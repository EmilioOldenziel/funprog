\documentclass[a4paper]{article}

\usepackage{a4wide,times}
\usepackage[english]{babel}
\usepackage{listings}
%\usepackage{mathtools}
\usepackage{amsmath}
\lstset{language=Haskell,tabsize=1, basicstyle=\fontsize{7}{8}\selectfont\ttfamily}

\begin{document}

\title{Programming report\\
       assignment Functional Programming
}
\date{\today}
\author{Emilio Oldenziel (s2509679)\\  Lotte Noteboom (s1959492)}

\maketitle

\section{Proof}
In this section we will proof that $a^{q-1} \: mod \: p = 1 $ with $n=p*q$ is true for each $q$ for which
$q-1$ is a multiple of $ord_{a}(p)$, thus 
$q-1 = k * ord_{a}(p) \implies a^{q-1} \: mod \: p = 1$
We will proof this with the reductio ad absurdum. Since $ord_{a}(p) = \operatorname*{min}_e a^{e} \: mod \: p =1$, we will change the right
hand side of this equation by adding an extra value $r$ with $r < ord_{a}(p)$, this will give us the following expression:
\[q-1 = x * ord_{a}(p) + r\]
This will lead to the evaluation below.

\[a^{q - 1} \equiv a^{x*ord_{a}(p) + r}\]
\[a^{q - 1} \equiv (a^{ord_{a}(p)})^{x} * a^{r}\]
\[a^{q - 1} \equiv 1 * a^{r} \equiv a^{r} \not\equiv 1\]

Because the evaluation ends in a false conclusion, the first statement is true, no matter what we fill in for $r$ this
this expression will never be true, thus $a^{q-1} \: mod \: p = 1 $ is true for each $q$ where $q$ is a multiple of
$ord_{a}(p)$.

\section{Excersises}
\begin{tabular}{c|c}
1a & isPrime\\
1b & cntPrime\\
1c & oddPspTO\\
1d & expmod\\
1e & oddPspTOI\\
\\
2a & order2\\
2b & oddPspTOII\\
2c & isPrime2\\
2d & cntPrime2\\
\end{tabular}
\newpage
\section{Program text}
\lstinputlisting{opdr1.hs}

\end{document}